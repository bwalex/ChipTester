\documentclass[journal]{IEEEtran}

% correct bad hyphenation here
\hyphenation{op-tical net-works semi-conduc-tor}

\providecommand*\email[1]{\href{mailto:#1}{#1}}

\usepackage{color}
\usepackage{soul}
\usepackage{graphicx}
\usepackage{amsmath}
\usepackage[center]{caption}
\usepackage{multirow}

\begin{document}
%
% paper title
\title{VLSI Design Project \\%
ELEC6027 \\%
\medskip
\Large{Individual Reflection and Evaluation Report} \\%
\medskip
\large{GDP Team Number: \emph{Team I}}  \\%
\large{Title: \emph{Automated Digital Chip Tester}}  \\%
\smallskip
\large{Author's name: \emph{Pavlos Progias}}  \\%
\smallskip
\large{Supervisor(s): \emph{Dr Peter R. Wilson}}  \\%
}
%
%
% author names and IEEE memberships
% note positions of commas and nonbreaking spaces ( ~ ) LaTeX will not break
% a structure at a ~ so this keeps an author's name from being broken across
% two lines.
% use \thanks{} to gain access to the first footnote area
% a separate \thanks must be used for each paragraph as LaTeX2e's \thanks
% was not built to handle multiple paragraphs
\author{}% <-this % stops a space
% \thanks{Manuscript received January 20, 2002; revised November 18, 2002.
%         This work was supported by the IEEE.}% <-this % stops a space
% note the % following the last \IEEEmembership and also the first \thanks - 
% these prevent an unwanted space from occurring between the last author name
% and the end of the author line. i.e., if you had this:
% 
% \author{....lastname \thanks{...} \thanks{...} }
%                     ^------------^------------^----Do not want these spaces!
%
% a space would be appended to the last name and could cause every name on that
% line to be shifted left slightly. This is one of those "LaTeX things". For
% instance, "A\textbf{} \textbf{}B" will typeset as "A B" not "AB". If you want
% "AB" then you have to do: "A\textbf{}\textbf{}B"
% \thanks is no different in this regard, so shield the last } of each \thanks
% that ends a line with a % and do not let a space in before the next \thanks.
% Spaces after \IEEEmembership other than the last one are OK (and needed) as
% you are supposed to have spaces between the names. For what it is worth,
% this is a minor point as most people would not even notice if the said evil
% space somehow managed to creep in.
%
% The paper headers
% \markboth{Journal of \LaTeX\ Class Files,~Vol.~1, No.~11,~November~2002}{Shell \MakeLowercase{\textit{et al.}}: Bare Demo of IEEEtran.cls for Journals}

\markboth{ELEC6027 - Individual Reflection and Evaluation Report - \MakeLowercase{pp5g11}}{}
% The only time the second header will appear is for the odd numbered pages
% after the title page when using the twoside option.

% make the title area
\maketitle


\begin{abstract}

In this report, there is an abstract.

\end{abstract}

% \begin{keywords}
% IEEEtran, journal, \LaTeX, paper, template.
% \end{keywords}
% Note that keywords are not normally used for peerreview papers.

% \section{Introduction}

\PARstart{H}{owever} there is also some text. First goes an introduction.


\section{Group Formation and Operation}
\label{sec:formation}

How did the group form?  How well did it function?  What role(s) did you and the other group members play?  Were there any gaps, or problems determining roles?

Grade your group /10



\section{Work Breakdown}
\label{sec:work_breakdown}

What were your group’s strengths and weaknesses?  How did you allocate responsibilities to different members of the group?  Did you all contribute equally, or did some people do more work than others?

Grade your group’s work breakdown /10




\section{Planning and Progress}
\label{sec:planning}

How did the group plan the project?  Which suggestions were yours?  To what extent did you manage to follow the plan?  What adjustments did you have to make?

Grade your group's planning \& progress /10




\section{Your Own Contribution}
\label{sec:own_contribution}

Considering the aspects of the project for which you had responsibility, justify the decisions you made, and assess how effective your contributions were.  

Grade your own contribution /10



\section{Reflective Evaluation}
\label{sec:reflective_evaluation}

What lessons did you learn?  With hindsight, would you do anything differently?






\appendix

You should include in the appendix photocopies or extracts from your log-book and email logs in support of your statements.  In case of serious discrepancies between different accounts, the examiners may request complete log-books and email logs.



% \begin{figure}[h!]
% \centering
% \includegraphics[width=.9\columnwidth]{figures/leach_move_p3}
% \caption{Sequence of frames showing movement of particle in liquid because of the rotation of the vaterite particles forming the pump described by J. Leach et al. in \cite{Leach2006}. Figure taken from \cite{Leach2006}.}
% \label{fig:leach}
% \end{figure}




\section{Discussion and Conclusions}
\label{sec:conclusions}

And in the end, there will be some discussion and conclusions.


% \bibliographystyle{IEEEtran}
% \bibliography{mems_rf}


\end{document}


