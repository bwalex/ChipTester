% This report should be at most 2,000 words, around 5 pages.  You are advised to include the following sections, or similar.   You are asked to write honestly about how the project went, and to grade various aspects on a scale of 0 (hopeless) to 10 (perfect).  Your comments will be kept confidential from other members of the group.

\section{\texorpdfstring{\href{mailto:pp5g11@ecs.soton.ac.uk}{Pavlos Progias}} {Pavlos Progias}}

\subsection{Group Formation and Operation}
\label{sec:formation}

% How did the group form?  How well did it function?  What role(s) did you and the other group members play?  Were there any gaps, or problems determining roles?
%
% Grade your group /10

The group of team I was well organised and functioned well and concisely throughout the semester. There were never any problems in communication between the members, and everyone did their best to contribute as much as possible, without staying behind on their other modules as well.

The members of their group all had slightly different preferences or experience in work area, making it easier for the work to be split accordingly between the members.

Regular meetings held almost weekly (in many cases more than once a week, when needed) that all members attended, ensured that the team was making steady progress. In these meetings, each team member shortly presented their progress and discussed any problems they had come across. This way all team members would keep up with each other's progress and could ask for help if needed.

One of the group members, Alex, was more experienced and knowledgeable in the field of FPGA implementations and from the start of the project was unanimously considered to be the ``coordinator'' of the group. Because of his industry experience in hardware design he could help a team member had they come across problems in their own work, and also organised and put together the various components that would form the final design.

I would grade the group with a $10/10$, since albeit the differences in experience and level of knowledge on relevant subjects, the team members showed good cooperation with no communication problems and efficient planning.











\subsection{Work Breakdown}
\label{sec:work_breakdown}

% What were your group’s strengths and weaknesses?  How did you allocate responsibilities to different members of the group?  Did you all contribute equally, or did some people do more work than others?
%
% Grade your group’s work breakdown /10

In my opinion the only problem in the group of Team I, which boosted rather than hindered the progress of the group as far as the project was concerned, but only introduced inequalities in work load, was the noticeable differences in experience and detailed knowledge in materialising such a project on FPGAs between the group members.

Differences in experience and fluency on aspects that were key to the success of the project resulted in large tasks being handled by group members on their own. This definitely contributed significantly to the success of the project, but created an uneven distribution of labour, although all team members tried their best to offer as much as possible to the team and make the best out of the allocated project. Had the module been structured differently and planned in advance, with lecture sessions and help on using the tools and resources needed, this could have been avoided since it would be more likely for all team members to able to contribute equally, albeit the limited time frames.

Under the current circumstances, I consider my group to have done a good job, cooperating well and trying to gain as much knowledge as possible in order to help share the workload and contribute to the progress of the design project. This ensured the steady progress of the team and friendly and productive interactions between the team members.

The allocation of labour was done as the project was progressing, starting from basic tasks that had to be completed first to provide the backbone of the project (such as the implementation of the Nios-II core implementation, the design of other key modules such as the frequency counter, the design of the interface PCB and the implementation of the database server). As progress was being made, more minute tasks were assumed by team members depending on their skills and preferences (such as implementing additional functionality on the server or providing a user-friendly design for the interface website).

Because of the continuous allocation of tasks to team members, everyone was working on some aspect of the project most of the time, however with some tasks being completed quicker than others in some cases, it is certain that not all team members carried out the same amount of work. In particular, Alex worked on more parts of the project and would also help the rest when they had trouble completing their parts. This was to be expected, though, and can be mostly attributed to the lack of lectures and guidance to help the progress of the teams.

I would grade the work breakdown in the team with $9/10$ since it was effective, resulting in a good completed project but was not perfect. On one hand, team members ended up doing different amounts of work, but on the other hand the distribution of labour was efficient and everybody contributed to the team as much as possible, working on the aspects they preferred and were more knowledgeable at as much as possible.




\subsection{Planning and Progress}
\label{sec:planning}

% How did the group plan the project?  Which suggestions were yours?  To what extent did you manage to follow the plan?  What adjustments did you have to make?
%
% Grade your group's planning \& progress /10

The project was initially broken down in parts, such as the interface to the chip under test, the user interface or the processor core. For each part, a rough overview was made of the required functionality and each team member chose to start working on one aspect of the project, depending on their preferences. Once the main tasks were taken care of, interfacing between the modules and secondary parts of the design were tackled.

The group managed to stick to the plan, although no specific deadlines were set much in advance. However, the work was carried out in time and the project was completed without sacrifices or skipped steps to make the project deadline.

In order for the project files to be accessible by all the team members, to monitor changes and also be able to have access to the latest versions, a github project was set up. All team members had their computers synced with the git project and once changes were made to pieces of code the author would upload them to the git project, including comments with what was changed. This worked out well, since backups of everything were kept all the time, minimising the risk of data loss, but also because all team members could access other parts of the project to use as references or paradigms.

To keep communication between team members, emails were generally addressed to all of the team members, to arrange for meetings and to update with recent work and progress. That way, everybody could confirm they had no other obligations and could attend all the meetings.

Taking all of the above into consideration I would grade the team with a $10/10$, since the project was completed within the time allocated and no unexpected situations having appeared within the semester. The team worked more or less according to plan, working efficiently, cooperating well and helping each other to avoid delaying when needed.


\subsection{Your Own Contribution}
\label{sec:own_contribution}

% Considering the aspects of the project for which you had responsibility, justify the decisions you made, and assess how effective your contributions were.
%
% Grade your own contribution /10

In the parts of the project I was responsible for I started with overviews of what the final outcome should be like and elaborated on that. The parts I mainly worked on were the frequency counter module of the design, the layout of the website and the compilation of the report in latex format.

For the frequency counter, I started with a simple frequency counter design that would count the period of an input signal. The basic design was then worked on to include enabling signals and local memory. In the end, the interface of the design was worked on to be able to communicate efficiently with the rest of the superchip tester, following the specifications provided set by the team.

As far as the design of the website is concerned, the database management and the basic functionality of the webpage was taken care of by another team member. The website, however, was very simple and not user-friendly. In order to be efficient and not waste time that I could spend on other parts of the project, I decided to start with a website template that would look good in the first place and build the functionality needed for my team's website into it. I integrated the simple but functional web interface my team had already designed into a more elaborate and well-designed website and built the extra functionality needed into that. The original template was drastically changed to adapt to the needs of our applicatio and provide the required functionality and ease of use. From then on, when new functionality would be added into the web interface, I took care of aligning its look and feel with the rest of the webpage.

The report was formatted in latex, because of its better presentation, efficiency and modularity. I converted parts of it that were not written in latex in the first place, trying to format it to provide consistency throughout the report. In the end, I read through the report again, trying to identify and fix problem areas such as spelling mistakes, figure formatting problems etc.

I believe my work in the team contributed to the overall project effectively, either providing functionality that the chip tester needs to include, or enhancing the quality and user-friendliness of the web interface, to ensure my team's work does not seem sloppy and not well-made.

I would grade my contribution to the team with a $9/10$, since I did my best to help as much as possible, but my lack of experience in certain fields meant that I learned a lot, but could not offer as much as other team members.



\subsection{Reflective Evaluation}
\label{sec:reflective_evaluation}

% What lessons did you learn?  With hindsight, would you do anything differently?

Working for this project taught me a lot regarding FPGA manipulation, providing me with practical insight on FPGA programming. The whole process of designing, fabricating and testing a whole system that meets certain specifications has been followed.

My team working skills were also enhanced. In order for the project to succeed, teams had to cooperate well, allocating their members to tasks they were most familiar or efficient with. The use of project management tools such as the github was also a very useful aspect of the project, providing me with practical experience on using such tools.