\chapter{Individual Reflexion. Romel Torres (rt5g11@soton.ac.uk)}

\section{Group Formation and Operation}

The group was randomly chosen by Peter Wilson, Alexander Hornung, Xiaolong Li, Pavlos Progias and Bo Zhou were my team-mates, I personally consider that Alexander 
took the role of project manager and did an excellent job, the rest of us tried to keep up and work as much as possible in the task we each took. The group functioned 
quite well, despite its multiculturalism as it would be in a professional environment, each member did the tasks assigned to them and tried to have them done by the set
 up deadlines. There were not many problems between us, we were all present in almost every meeting we had, there were few if any communication problems or misunderstandings 
among us. 

Nevertheless, I think that we should have more “work meetings” (where we gathered and worked together) instead of usually having “progress meetings” because most
 of our work was done individually and then uploaded to the repository in GitHub; this way we would have learned much more about each others parts and could have even 
worked faster. In the progress meetings we usually discussed about next steps and ideas to implement them and then we implemented the results and discussed them by emails.

Grade your group: 8/10

\section{Work Breakdown}
I personally think that the group was well equipped: We are all hard working and tried to do our best in every task we undertook; Alexander had a vast experience 
in Linux and C, Xiaolong and Bo had some experience in Hardware design, Pavlos had some experience with HTML and CSS design and I had some experience with web 
server development. 

We allocated the responsibilities by experience associated to the task, Alexander worked on the SoPC and ucLinux software, Xiaolong and Bo worked on the external 
hardware and modules for the SoPC, Pavlos worked in the user interface and SoPC modules and I worked on the  ba-ckend and modules for the Linux software. We all 
tried to contribute as much as possible in the project, nevertheless, undoubtedly Alexander did more work than the rest of us. Due to his experience and programming 
skills was able to do in just minutes work that for the rest of us would take hours of learning and debugging. We all tried to learn as much as possible from him 
and do our tasks as fast and good as possible, the rest of us contributed more or less equally to the project doing all we set up to have.

Grade your group’s work breakdown: 8.5/10

\section{Planning and Progress}

We came up with a Gantt chart around the second meeting and stuck to it until the 8th or 7th week of the project. Afterwards we diverged from the original plan 
because some new ideas for the project came up and new decisions were made.  We managed to follow the plan up to certain limits, but at the end we had to rush a 
bit to complete everything. 

The way we managed the project using the repository was of enormous help. Git is an excellent tool for managing software projects and it assured that everybody 
had a working copy of the project at all times. 

The project advanced quickly until the Easter period when some of us went out of the country (including me) and the project advanced slower during that period, 
afterwards it advanced quickly again and we managed to step by step build the whole application.

We had to make few adjustments due to the fact that the boards we built arrived earlier than planned (external causes) nevertheless that did not affect the 
back-end development. On the other hand for the flash programmer a model of the flash was created and the program was tested with it, this allowed us to debug 
the code independently of when the components arrived.

Grade your group’s planning and progress:	7/10

\section{Own contributtion}

I had the responsibility of the back-end, I initially suggested doing it in Java and even did some classes to connect to the database. Nevertheless, we decided 
to switch to Ruby (even though I had never worked on it) because Ruby was designed to develop web applications quickly and with Java would take much longer. The 
idea of using a scripting language such as PHP, Python or Ruby was Alexander's idea, but we decided to go for Ruby and turned out to be a great language, most of 
the oriented programming skills I had translated easily to it, and the back-end was developed smoothly. 

In the back-end I developed most of the classes that translate into the database, all the security related to the management of password and user sessions, the uploading 
files to the server, the initial APIs for communicating to the board, the database management, the email sending and display of the results in the back-end.  

I also contributed by writing the SPI software that controls the flash memory that goes into the second FPGA, I also initially developed the parsing program for the test 
vector but was later re-factored. 

Grade your own contribution: 7/10 

\section{Reflective Evaluation}

I personally learned a lot of code management using Git, which is a powerful tool in nowadays software development world. I learned a new programming language: ruby 
and also about web server development (both for serving HTML content and API for embedded systems). I learned also a lot about linux for embedded systems and C for 
embedded systems and communication using the HTTP protocol between a system able to send HTTP request (micro-controllers, FPGAS, computers,..) and web servers, this
is a quite relevant because today's world is dominated by the Internet.

Regrettably I did not work much with HDL for the FPGA and that would have been a good asset, nevertheless, in a project like this where so many areas are integrated,
we cannot all work on the same things and acquire the same skills, probably the most important lesson we have learned here, beyond the professional skills, is project
management and work integration with other teammates. The work undertook with all the teammates in a professional environment where everybody did their best in a 
professional way has taught me more about project management and I consider it as an invaluable experience for my future professional career.

I think that I would have gotten more involved in the custom hardware design of the FPGA board, because I think that even though I know SystemVerilog and SystemC I need 
more practice in the languages.    
