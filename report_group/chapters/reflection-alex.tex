\documentclass[a4paper,10pt]{article}
\usepackage[utf8x]{inputenc}

%opening
\title{}
\author{}

\begin{document}

%XXX: also add grades
\section{Individual Reflection and Evaluation}
\subsection{Group Formation and Operation}
% How well did it function?  What role(s) did you and the other group members play?
% Were there any gaps, or problems determining roles?
The group functioned well overall. The target was achieved and the requirements
were met. As a team the group had enough experience in every relevant area of
the subject to be able to successfully complete the project without any particular
gaps of knowledge.

However I felt that not all team members were happy with the group operation, particularly
due to parts of how the work breakdown was done.

Grade: 7 / 10


\subsection{Work Breakdown}
% What were your group’s strengths and weaknesses?  How did you allocate
% responsibilities to different members of the group?  Did you all contribute
% equally, or did some people do more work than others?
Responsibilities were generally allocated taking into account the strengths
and weaknesses of the individuals, and also personal interest.

None of the responsibilities were directly assigned by any one individual, but
all were distributed by someone volunteering to do a given task.

However personal interest in a given subject was not always taken into account
when for example another team member had more experience in the same area. In this
case, due to the limited amount of time, the task was generally allocated to
the more experienced team member rather than the most interested/eager one.

This also resulted in some team members doing a larger amount of work than others
in the overall view of things. However none of the un-evenness in work distribution
stemmed from problems with any one team member; everyone was eager to work and
participate and contribute to the project.

I have the impression this type of allocation caused some unhappiness amongst
some of the team members, but it allowed us to meet the target on time.

Grade: 6 / 10



\subsection{Planning and Progress}
% How did the group plan the project?  Which suggestions were yours?  To what
% extent did you manage to follow the plan?  What adjustments did you have to make?
Weekly meetings during the project except for the easter break allowed us to keep
track of progress throughout the project.

Particularly during the Easter Break contributions by all team members
dropped significantly. Although expected, this resulted in having to work on
a tigher time scale during the late stages of the project.

The drop in activity can be blamed on lack of detailed planning and progress
reporting/progess meetings to keep track of what everyone is working on.

Overall we generally managed to stay on schedule. However we made some late
amendments to the project targets, including dropping support for a more full-blown
oscilloscope in favour of a much simpler ADC capturing mechanism. This allowed
us to add a new target: virtual (pre-fabrication HDL) testing.

A particularly problematic area has been the hardware and component sourcing. While
we submitted a list of required materials very early on, we did not receive them
until almost two months later, during the Easter Break. This resulted in us having
to first delay prototyping, then finally scrap it completely. Also problematic
was getting our hands on a footprint for the HSMC connector for our PCBs. It took
us several weeks to get a proper PCB footprint for the HSMC connector from the manufacturer (Samtec)
itself. This further delayed the whole hardware and PCB part and ended up pushing
it into the last week(s) of the project.

Grade: 6 / 10



\subsection{Own Contribution}
% Considering the aspects of the project for which you had responsibility,
% justify the decisions you made, and assess how effective your contributions were.
As a late addition to the team I came in to a team which had already laid
out a plan of how to tackle the problem. However, I felt that the approach
the project was taking would have resulted in a much larger effort than needed.
The initial plan involved doing everything in hardware, including reading configuration
from the SD cards, the ethernet communication, etc.

When I joined the team I laid out a new plan breaking up the effort into both
a hardware and a software component. The software component would be implemented
using an embedded processor on the FPGA. This new plan was the actual plan followed
by the team for the rest of the project. The overall design of the embedded software,
FPGA hardware and partly the backend was also done by me.

I also served as liaison for the team with external people such as lecturers,
manufacturers, etc. Effectively I acted as the team lead, ensuring everyone
had something to do, managing resources, meetings, etc.

I implemented the configuration reader embedded software, the Linux drivers, the SoPC,
and the tester module. I also reviewed the work by all other team members,
often coming up with suggestions on improvements.

I also took took care of the final integration of most the parts and the development
of an easier-to-use build infrastructure for the project and its external resources.

Grade: 7 / 10





\subsection{Reflective Evaluation}
% What lessons did you learn?  With hindsight, would you do anything differently?
While the project overall was successful, several areas could have been improved
upon in hindsight.

The fact that most of the overall design/architecture was done by me resulted
in a loss of perspective for other team members, even though we held weekly sessions
in which we would go through and discuss the design. In hindsight the design
process should have been more open, with everyone coming up with a design and then
collating the results.

Overall I also took on too much work, leading to an uneven work distribution. As
mentioned earlier, this type of work distribution lead to some felt unhappiness in the
team.

The slow progress during the Easter Break can also be traced back to me, as I acted
as a team lead. I should have pushed for regular meetings even during this
period.

I realize that it was
a pilot project, but nonetheless the management by the course leaders has not been good enough. All
scheduling by part of the lecturers was done last second, and last minute additions
and changes to scheduling and guidelines caused confusion.

The scheduling of the final presentation in particular has been abysmal, making
it fall directly into the exam period. My final presentation for example is just
a few hours after an exam.

The PCB fabrication has also been a problematic area. We had discussed and arranged
for a quick turnaround PCB fabrication. However the person in charge (a post doctoral
student) forgot to send the PCBs off for fabrication on the agreed date. This resulted
in us receiving the PCBs only the day before the final report was due. Similarly
the first round of component sourcing took almost two months, delaying part of
the project as previously mentioned.

While I learned a great deal with this project and gained experience in a range
of subjects, I feel that the course module itself was mismanaged.







\end{document}
