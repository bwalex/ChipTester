\chapter{Embedded Software}
------------------------

\section{Bootloader and Operating System}
To take full advantage of the MMU, Linux was chosen as operating system running on the
application processor. uCLinux was the distribution of choice since it is aimed at
embedded systems and optimum resource usage. A Linux 3.3-rc6 kernel was used.
\\

Since the Linux kernel can only boot from memory, a bootloader was required to
load the Linux kernel image from the non-volatile CFI Flash into SDRAM. The bootloader
used is a current development version of U-Boot.
\\

\subsection{CFI Flash Layout}
The CFI Flash was partitioned into several MTD partitions. Table \ref{table:partitions}
shows a map
of the MTD partitions and their respective size.

\begin{table}[h!]
\centering
\begin{tabular}{ | l | r | l | }
 \hline
   Address Range       & Size (kB) & Description \\
 \hline
   \texttt{0x00000000 - 0x0003ffff} & 256  & U-Boot \\
 \hline
   \texttt{0x00040000 - 0x0004ffff} & 64 & U-Boot configuration \\
 \hline
   \texttt{0x00050000 - 0x001fffff} & 1728 & Linux Kernel Image \\
 \hline
   \texttt{0x00200000 - 0x0077ffff} & 5632 & Linux Root Filesystem \\
 \hline
   \texttt{0x00780000 - 0x007fffff} & 512 & Misc \\
 \hline
\end{tabular}
\caption{Partition Layout of CFI Flash Memory}
\label{table:partitions}
\end{table}

The U-boot bootloader image is
located in the first 256 kB, of which around 170 kB are used by u-boot. The CPU
reset vector points to this address, so that u-boot always is loaded on CPU reset.
The configuration for U-boot that contains information on how to boot the Linux system
is located at the next 64 kB. Only 318 bytes are currently used.
\\

The compressed Linux kernel image is located just after the u-boot configuration. The actual
size currently used by the Linux kernel is around 1200 kB. This image is loaded into memory
and decompressed by U-boot upon startup.
\\

The root filesystem of the Linux system is located in the largest partition. It is a JFFS2
file system containing the complete uClinux userland. It currently occupies around 4000 kB.
\\

The last partition is currently unused.


\newpage
\subsection{Bootloader}
The u-boot bootloader is located at the first 256 kB of the CFI Flash. The CPU's
reset vector points to it, so that upon startup of the CPU, u-boot is executed.
\\

U-boot prompts the user to interrupt automatic boot during a 5 second countdown. If a
user interrupts this process by connecting via UART and pressing a key, the u-boot
prompt appears. Using this prompt u-boot's configuration can be modified and stored.
\\

When booting automatically, U-Boot reads the configuration of how to boot from the
u-boot configuration on the flash. U-Boot will first read and decompress the Linux
kernel into memory and then pass control to it.
\\

A total of 3 small patches including bugfixes were submitted upstream to the U-Boot development
community to address a number of issues. 2 of the 3 patches were committed. One of the patches
fixed a bug in the memory allocation\footnote{http://lists.denx.de/pipermail/u-boot/2012-February/118453.html},
another implemented some timer functions\footnote{http://patchwork.ozlabs.org/patch/142141/} and the last
improved the logic around choosing JTAG UART or UART on the Nios II platform\footnote{http://patchwork.ozlabs.org/patch/142142/} but was not accepted.
\\


\subsection{OS}
A development snapshot of the uClinux distribution and Linux 3.3.0-rc6 kernel is used as the
application operating system. uClinux is a distribution aimed at embedded systems with a low
footprint. It replaces the GNU glibc with the much smaller uClibc and several core unix tools
with the minimalistic busybox toolsuite.
\\

In its current configuration, the kernel uses up around 1.2 MB of space, while the root filesystem
weighs in at around 4MB with a JFFS2 filesystem. Both are compressed.
\\

The booted system used only 10 MB of physical memory, leaving almost 120 MB to running programs. No
swap/paging space was configured, but since the MMU is enabled, program text pages are paged in
on demand and backing physical memory pages are also only allocated on access.
\\

A patch fixing an issue with SD Card interfacing over SPI was submitted upstream to the Linux
MMC team\footnote{http://comments.gmane.org/gmane.linux.kernel.mmc/12835}.
\\

Due to a bug in the linux kernel in the handling of variable block-size CFI Flash chips,
some of the MTD partitions on the CFI Flash are forced to be read-only, even though they
are aligned to valid sector boundaries.
\\

Another limitation of the Linux system as is is that it's not possible to use the on-board
USB chip ISP1632 in USB device mode, so it's not possible to connect the board running Linux
as a peripheral to a host machine.
\\

The boot log of the embedded system including both the u-boot bootloader and the Linux system
can be seen in Figure \ref{listing:bootlog}.

\begin{figure}[h!]
\lstset{basicstyle=\tiny\ttfamily}
\begin{lstlisting}
U-Boot 2011.12-00465-ge37ae40-dirty (May 14 2012 - 20:20:10)

CPU   : Nios-II
SYSID : abcdef00, Tue May 15 16:25:13 2012
BOARD : my_nios2

Hit any key to stop autoboot:  0
## Booting kernel from Legacy Image at ea050000 ...
   Image Name:   Linux-3.3.0-rc6
   Image Type:   NIOS II Linux Kernel Image (gzip compressed)
   Data Size:    1262781 Bytes = 1.2 MiB
   Load Address: c0000000
   Entry Point:  c0000000
   Verifying Checksum ... OK
   Uncompressing Kernel Image ... OK

Linux version 3.3.0-rc6 (alex@alex-pc) (gcc version 4.1.2) #25 We49.47 BogoMIPS (lpj=98944)
pid_max: default: 32768 minimum: 301
Mount-cache hash table entries: 512
gpiochip_add: registered GPIOs 246 to 246 on device: /sopc@0/bridge@0xb000000/gpio@0x220
gpiochip_add: registered GPIOs 245 to 245 on device: /sopc@0/bridge@0xb000000/gpio@0x200
gpiochip_add: registered GPIOs 244 to 244 on device: /sopc@0/bridge@0xb000000/gpio@0x210
gpiochip_add: registered GPIOs 243 to 243 on device: /sopc@0/bridge@0xb000000/gpio@0x230
JFFS2 version 2.2. (NAND)  2001-2006 Red Hat, Inc.
msgmni has been set to 241
Block layer SCSI generic (bsg) driver version 0.4 loaded (major 254)
io scheduler noop registered
io scheduler deadline registered (default)
ttyAL0 at MMIO 0xd100000 (irq = 11) is a Altera UART
console [ttyAL0] enabled, bootconsole disabled
ttyJ0 at MMIO 0x80034b0 (irq = 2) is a Altera JTAG UART
altera_sysid ba00000.sysid: System creation hash ABCDEF00 timestamp 2012-05-15 16:25:13
Test Runner Module loaded
trunner d000000.trunner: Test Runner device 0, irq 7
DE2LCD Module loaded
de2lcd b000010.de2lcd: DE2 LCD 0
Frequency Counter Module loaded
fcounter d200000.fcounter: Frequency Counter device 0, irq 12
ADC Module loaded
adc d300000.adc: ADC device 0, irq 13
a000000.flash: Found 2 x8 devices at 0x0 in 16-bit bank. Manufacturer ID 0x00703a Chip ID 0x000001
NOR chip too large to fit in mapping. Attempting to cope...
Amd/Fujitsu Extended Query Table at 0x0040
  Amd/Fujitsu Extended Query version 1.3.
number of CFI chips: 1
Reducing visibility of 16384KiB chip to 8192KiB
5 ofpart partitions found on MTD device a000000.flash
Creating 5 MTD partitions on "a000000.flash":
0x000000000000-0x000000040000 : "u-boot"
0x000000040000-0x000000050000 : "u-boot_cfg"
mtd: partition "u-boot_cfg" doesn't end on an erase block -- force read-only
0x000000050000-0x000000210000 : "kernel"
mtd: partition "kernel" doesn't start on an erase block boundary -- force read-only
0x000000200000-0x000000780000 : "rootfs"
0x000000780000-0x000000800000 : "config"
spi_altera ba10000.spi: base eba10000, irq -6
spi_altera ba20000.spi: base eba20000, irq 8
altera_tse-mdio: probed
eth0: Altera TSE MAC at 0xe8003000 irq 4/3
eth0: Reporting available PHYs:
eth0: PHY with ID 0x1410cc2 at 0x10
Clock not ready after 100ms
Initializing USB Mass Storage driver...
usbcore: registered new interface driver usb-storage
USB Mass Storage support registered.
mmc_spi spi32766.0: SD/MMC host mmc0, no DMA, no WP, no poweroff, cd polling
TCP cubic registered
NET: Registered protocol family 17
Freeing unused kernel memory: 3292k freed (0xc0218000 - 0xc054f000)
Preparing GPIO
Starting syslogd
Starting dhcpcd
Starting crond
Welcome to Team I's
   _____ _     _    _______       _
  / ____| |   (_)  |__   __|     | |
 | |    | |__  _ _ __ | | ___ ___| |_  ___ _ __
 | |    | '_ \| | '_ \| |/ _ | __| __|/ _ \ '__|
 | |____| | | | | |_) | |  __|__ \ |_|  __/ |
  \_____|_| |_|_| .__/|_|\___|___/\__|\___|_|
                | |
                |_|



BusyBox v1.18.4 (2012-05-15 19:19:33 BST) hush - the humble shell
Enter 'help' for a list of built-in commands.

root:/>
\end{lstlisting}
\caption{Boot log of system}
\label{listing:bootlog}
\end{figure}


\subsection{Device Tree}
The current generation of Linux kernels can use so-called device tree files to specify the address range,
interrupts and several other options of available peripherals. A device tree file is simply a list
of all the modules connected to the system and their configuration. It is possible to specify each module's
type, memory address range, interrupts, etc. Figure \ref{listing:dts} shows an excerpt of a devicetree file showing the
configuration for an Altera SPI master, an SD card connected to it, and the custom test runner module.
\\

The memory mapped address range is specified using the \texttt{reg} keyword, the connected interrupt numbers
are specified using \texttt{interrupts} and \texttt{interrupt-parent}. The \texttt{compatible} keyword is
used to identify the device type so that a given driver with the same compatibility can attach to it.

\begin{figure}[h!]
\lstset{basicstyle=\scriptsize\ttfamily}
\begin{lstlisting}
spi_0: spi@0xba10000 {
  compatible = "ALTR,spi-11.1", "ALTR,spi-1.0";
  reg = < 0x0BA10000 0x00000020 >;
  interrupt-parent = < &cpu >;
  interrupts = < 0 >;
  #address-cells = < 1 >;
  #size-cells = < 0 >;

  mmc_spi@0 {
    compatible = "mmc-spi-slot";
    spi-max-frequency = < 10000000 >;
    reg = < 0x00000000 >;
    voltage-ranges = < 3300 3300 >;
  }; //end mmc_spi@0
}; //end spi@0xba10000 (spi_0)

test_runner_0: trunner@0xd000000 {
  compatible = "trunner,trunner-1.0";
  reg = < 0x0D000000 0x00000100 >;
  interrupt-parent = < &cpu >;
  interrupts = < 7 >;
}; //end trunner@0xd000000 (test_runner_0)
\end{lstlisting}
\caption{Example excerpt from a devicetree file}
\label{listing:dts}
\end{figure}



\newpage
\section{Build Infrastructure}
As the project depends on several external resources such as the u-boot bootloader, Linux kernel
and uClinux distribution, a build system was developed to make it easy to integrate all these
parts into the main project.
\\

The complete build infrastructure consists only of standard unix Makefiles and shell scripts.
\\

All the external resources (uClinux, uClibc, Linux kernel, u-boot) are set up as git submodules.
They are pulled in without any changes at a particular revision from their upstream sources.
\\

To allow for custom patches and integrating custom files, two mechanisms are included in the build
system. One of them uses regular patches in the \texttt{patchq/} directory and applies them in
order to each of the submodules. The other mechanism takes the directories in \texttt{overlay/}
and overlays them on top of the submodules. This allows for a complete separation of the external
resources and the custom patches and files. The external resource can be reset to its default state
by yet another makefile target, which cleans out all non-versioned files. The patches and overlays
can then be applied cleanly. This also makes it trivial to update or downgrade the external
resource without the issues associated with patched files.


\newpage
\section{Drivers}
The SRAM is accessed directly from userland without a driver by simply memory-mapping the
region corresponding to the SRAM into the virtual memory of the running application via the
\texttt{mmap} system call.
\\

GPIOs are controlled via an interface in \texttt{/sys} that the Linux GPIO drivers expose. This
interface permits setting, clearing and reading pins by reading and writing from what seem
regular files.
\\

Four custom Linux drivers were developed for this project for the ADC, frequency counter, test
runner/controller and LCD modules. All four drivers share a common skeleton.
\\

All of them use the platform driver framework and detect their devices based on the devicetree
entries. The probe function remaps the IO range, sets up the interrupt (if needed) and
creates a character device node (which gets exposed in \texttt{/dev}). This character device
exposes the interface to the userland applications via a combination of \texttt{read},
\texttt{write}, \texttt{poll} and \texttt{ioctl} system calls on these nodes.
\\

The drivers interact with the hardware by reading from/writing to specific memory addresses
as specified in the register map for each device using the low level \texttt{ioread} and
\texttt{iowrite} routines.
\\

The drivers of the devices using interrupts (frequency counter, ADC, test runner) register
an interrupt handler which clears the interrupt register of the device by reading from it
and wakes up all threads sleeping on a given wait queue. This allows userland programs
to use the \texttt{poll} system call to wait for the device to complete its task without
polling and instead sleeping on the waitqueue. As soon as the interrupt handler is called,
the thread is woken up and execution continues.
\\

All device drivers use a \texttt{ioctl} interface to control the driver from userland.
The frequency counter and lcd driver in addition also expose a \texttt{read} or \texttt{write}
interface.

\subsection{Test Runner and ADC Drivers}
Both of these drivers are effectively the same with only minor differences such as the
address of the registers to match the differing register maps of the devices, and different
names of the created device nodes and the devicetree entries that they match.
\\

The test runner driver creates a device node \texttt{/dev/trunner0} while the adc
driver creates the device node \texttt{/dev/adc0}. Interaction with both is strictly
via \texttt{ioctl} and \texttt{poll} system calls only.
\\

The \texttt{poll} interface implemented is described above - \texttt{poll} will
sleep on a waitqueue until an interrupt wakes it up.
\\

The \texttt{ioctl} interface comprises only three messages. The \texttt{ADC\_IOC\_ENABLE} and
\texttt{TRUNNER\_IOC\_ENABLE} \texttt{ioctls} enable the peripheral by writing to the
enable register of the device.
\\

The \texttt{ADC\_IOC\_GET\_DONE}, \texttt{TRUNNER\_IOC\_GET\_DONE},
\texttt{ADC\_IOC\_GET\_MAGIC} and
\\
\texttt{TRUNNER\_IOC\_GET\_MAGIC} read the done and
ID (magic) registers of the devices, respectively.



\subsection{Frequency counter driver}
The frequency counter driver registers a character device node under
\\
\texttt{/dev/fcounter0}
with a \texttt{poll}, \texttt{ioctl} and \texttt{read} system call interface.
\\

The behaviour of \texttt{poll} and the \texttt{FCOUNTER\_IOC\_ENABLE} and
\\
\texttt{FCOUNTER\_IOC\_GET\_MAGIC}
\texttt{ioctls} is the same as for the test runner and ADC drivers.
\\

In addition to these, the frequency counter driver provides three additional \texttt{ioctls}.
The \texttt{FCOUNTER\_IOC\_SET\_IPSEL} \texttt{ioctl} writes to the input select register,
effectively choosing which of the input pins is multiplexed onto the frequency counting logic.
The \texttt{FCOUNTER\_IOC\_SET\_CYCLES} \texttt{ioctl} write to the cycle timeout count register. This
defines for how many cycles of the host clock the frequency counter will be counting edges on the
input signal.
\\

The \texttt{FCOUNTER\_IOC\_GET\_COUNT} \texttt{ioctl} and the \texttt{read} system calls both
read the counted edges register of the device. By knowing the host frequency and the ratio
between this edge count and the cycle timeout count, the frequency of the input signal
can be calculated.


\subsection{DE2 LCD Driver}
The DE2 LCD character driver registers a device node under \texttt{/dev/de2lcd0}.
This driver exposes both a \texttt{ioctl} and a \texttt{write} system call interface.
\\

Writing to the device node using the \texttt{write} system call will write the provided
text to the offset at which the \texttt{write} occurred. The LCD has 2 lines of 16 visible characters,
but each line is actually 40 characters long. Hence a write can be a maximum of 80 characters long.
\\

The driver exposes several \texttt{ioctl} commands. The \texttt{DE2LCD\_IOC\_CLEAR ioctl}
clears the LCD display of all characters. The \texttt{DE2LCD\_IOC\_CURSOR\_ON} and
\\
\texttt{DE2LCD\_IOC\_CURSOR\_OFF} commands turn the blinking cursor on and off, respectively.
The final command, \texttt{DE2LCD\_IOC\_SET\_SHL} enables and disables shifting of the
characters on the LCD. The parameter passed to this last command is the time between shifts
in ms, or 0 to disable shifting. When a non-zero value is passed in, the driver registers
a function to be called regularly by the kernel timers. This function shifts the display
left each time it is called. When a zero value is passed in the timer is deactivated again.
This effectively allows displaying all 40 characters on each line of the LCD instead of just
the 16 visible ones.




\newpage
\section{Configuration Format}
The configuration format used by the software is an easy to read, easy to write and very
lenient format.
\\

Configuration is loaded from an SD Card. The SD Card should have a directory layout
as shown in Figure \ref{listing:dir_layout}.
The top-level \texttt{chiptester} file can be empty and only
indicates to the system that this SD Card contains test vectors.
\\

The top level directory should include a subdirectory for each team. Each team needs
to provide at least a \texttt{team.cfg} file in their subdirectory. They can also provide
an arbitrary number of test vector files which need not follow any naming
convention (e.g. \texttt{foo.bar} is also a valid vector file).

\begin{figure}[h!]
\lstset{basicstyle=\scriptsize\ttfamily}
\begin{lstlisting}
.
|- chiptester
|-- team1
|   |-- adder.vec
|   |-- shifter.vec
|   |-- team.cfg
|-- team2
|   |-- divider.vec
|   |-- team.cfg
|-- team3
    |-- multiplier.vec
    |-- team.cfg
\end{lstlisting}
\caption{Example directory layout}
\label{listing:dir_layout}
\end{figure}

\subsection{team.cfg}
The \texttt{team.cfg} file has four different keywords (team, email,
academic\_year and base\_url, all except the team and the base\_url can be omitted).
These keywords, their arguments and their use are explained in Table \ref{table:teamcfg_keywords}.

\begin{figure}[h!]
\lstset{basicstyle=\scriptsize\ttfamily}
\begin{lstlisting}
team: 3
email: foo@bar.com, baz@bar.net
academic_year: 2011/12
base_url: http://192.168.0.10:4567
\end{lstlisting}
\caption{Example team.cfg}
\label{listing:teamcfg_example1}
\end{figure}



\begin{figure}[h!]
\lstset{basicstyle=\scriptsize\ttfamily}
\begin{lstlisting}
# This is a comment
  team: # This is also a comment
    3
  email:
    foo@bar.com, baz@bar.net

academic_year : 2011/12
base_url      : http://192.168.0.10:4567
\end{lstlisting}
\caption{Another example team.cfg}
\label{listing:teamcfg_example2}
\end{figure}


A keyword can be preceded by any number of whitespaces (tabs or spaces). It must be followed
by a colon, but it is valid to have an arbitrary number of whitespaces between the keyword
and the colon. After the colon any number of whitespaces and newlines can follow before
the keyword-specific content (argument). Another example shown in Figure \ref{listing:teamcfg_example2} shows most of these features
in use. The parser is however case sensitive.
\\

All characters after a hash (\#) are treated as comments.

\begin{table}[h!]
\centering
\begin{tabular}{ | l | l | p{6cm} | }
 \hline
   Keyword       & Arguments & Description \\
 \hline
   \texttt{team} & $<$decimal number$>$ & The team number; saved in the database \\
 \hline
   \texttt{email} & $<$string$>$ & The email addresses, comma separated, to which to send the results of the test \\
 \hline
   \texttt{academic\_year} & $<$string$>$ & The academic year; saved in the database \\
 \hline
   \texttt{base\_url} & $<$string$>$ & The URL to the Chip Tester web server. This is used internally to send results to the backend and database \\
 \hline
\end{tabular}
\caption{Valid team.cfg keywords, their arguments and meaning}
\label{table:teamcfg_keywords}
\end{table}

\newpage
\subsection{Test vector files}
The parser used for the test vector files is the same as for the \texttt{team.cfg}.
As a result, it shares the same leniency - keywords can be prefixed by whitespaces,
the colon after a keyword can be prefixed by whitespace, the argument(s) after a
keyword, colon, can follow after any number of whitespaces and newlines.
\\

Every line after a keyword that does not contain a keyword by itself is parsed
using the same parser as the previous keyword. In other words, every line following
a keyword is parsed by the same keyword specific parser unless a new keyword appears.
\\

A special kind of keyword can appear inside lines parsed by the \texttt{vectors} line
parser. These keywords begin with a dot (.) and are not followed by a colon. These
are parsed by a separate parser, but the default line parser is not changed, so that
the line following them is still parsed by the \texttt{vectors} parser.
\\

Table \ref{table:vec_keywords} describes all normal keywords while table
\ref{table:dot_commands} describes all dot commands.
Figure \ref{listing:testvec_example} shows an example test vector file.

\begin{figure}[h!]
\lstset{basicstyle=\scriptsize\ttfamily}
\begin{lstlisting}
design: 1-bit shift left
clock:     A23
frequency: 1
trigger:
           Q23, Q22
pindef:
        A3, A2, A1, A0, Q3, Q2, Q1, Q0
vectors:
        0001 0010
        0001 T 0010
        0010 0100
.measure adc
        0011 0110
        0100 0100
        0100 1000
.measure frequency Q1
        0101 1010
        0110 1100
        0110 110X
.measure frequency Q2
        0111 1110
        1000 0000
        1000 1000
        1001 0010
        0100 W5 0111
        0001 T 1010
\end{lstlisting}
\caption{Example test vector file}
\label{listing:testvec_example}
\end{figure}

This example file contains test vectors
for a design called ``1-bit shift left''. The design will be tested at 1 MHz, and
this 1 MHz clock will be connected to pin A23. The triggered test vectors
will complete when Q23 and Q22 are asserted. The pins used in the test vectors are
A3, A2, A1 and A0 for the input test vectors, and Q3, Q2, Q1, Q0 for the results.
\\

Most test vectors are simple 1-cycle test cases where the result one cycle after applying
the input should match the given output. It also contains two triggered test cases
which complete whenever the trigger condition of both Q23 and Q22 being asserted is met, or
alternatively, after the default timeout of 32 cycles. Another test checks that the
output 5 clock cycles after applying the vector \texttt{0100} is \texttt{0111}.
\\

This file also contains three measure commands. The first activates the ADC module and
stores the captured result to the backend. The following two measure the frequency on
pins Q1 and Q2 respectively. The default timeout of $2^{24}$ cycles is used so that with
the 100 MHz system clock the measurement completes in 0.16 seconds, giving a resolution
down in the tens of Hz.


\begin{table}[h!]
\centering
\begin{tabular}{ | l | l | p{7.5cm} | }
 \hline
   Keyword       & Arguments & Description \\
 \hline
   \texttt{design} & $<$string$>$ & A string describing the design to be tested (e.g. ``4-bit adder'') \\
 \hline
   \texttt{frequency} & $<$decimal number$>$ & The frequency at which the device is tested. Minimum is 1, maximum is 100. The unit is MHz. Default is 10 MHz. \\
 \hline
   \texttt{clock} & $<$pins$>$ & A list of (input) pins to which the clock signal is connected. \\
 \hline
   \texttt{trigger} & $<$pins$>$ & A list of (output) pins which need to be asserted (all of them) for a triggered test to complete without a timeout \\
 \hline
   \texttt{pindef} & $<$pins$>$ & An list of input and output pins for which vectors and expected results are provided in the vectors section \\
 \hline
   \multirow{3}{*} \texttt{vectors} & $<$binary digits or X$>$ & A list of binary digits making up a test vector and expected result. first digit will be applied or checked against the first pin in the \texttt{pindef} and so on. An X can be specified on an output pin, meaning that that pin is ignored when checking the result of that particular test vector \\
    & or T[$<$number$>$]  & \texttt{T} specifies that this is a triggered test case which only completes after either the triggers specified with \texttt{trigger} are matched or a timeout, which by default is 32 cycles but can be overriden by providing a cycle number immediately after the T. \\
    & or W$<$number$>$ & \texttt{W} specifies that this is a fixed latency test case, whose result will be checked after the number of cycles specified. If neither T nor W are specified, then the implicit default is equivalent to \texttt{W1} \\
 \hline
\end{tabular}
\caption{Valid keywords, their arguments and meaning}
\label{table:vec_keywords}
\end{table}


\begin{table}[h!]
\centering
\begin{tabular}{ | l | l | p{5cm} | }
 \hline
   Keyword       & Arguments & Description \\
 \hline
   \texttt{.measure adc} & (none) & The ADC measure command will enable the ADC module after the previous vector has finished and send the results of the capture to the server \\
 \hline
   \texttt{.measure frequency} & $<$pin number$>$ $<$timeout$>$ & This command enables the frequency counter on a given pin number (either just a decimal number, or a decimal number prefixed by Q) for a given number of cycles and stores the result to the server\\
 \hline
\end{tabular}
\caption{Valid dot commands}
\label{table:dot_commands}
\end{table}


\newpage
\section{confrd}
The confrd program is the main application running on the embedded Linux system. It
is written in C for performance and memory footprint reasons.
It parses configuration structures such as the one outlined in the previous section, parsing
each file and controlling the peripherals such as the test runner, frequency counter and
ADC module.

\begin{figure}[h!]
\lstset{basicstyle=\scriptsize\ttfamily}
\begin{lstlisting}
 Usage: confrd [options] <configuration directory>
Valid options are:
 -p
         Print out the data written to SRAM in a human readable form
         on screen.
 -s <file>
         Write an SRAM initialization file containing the data generated
         using the configuration file(s).
 -v
         Marks this as a virtual design. Only affects remote logging.
 -w
         Write to actual SRAM and start the test runner after reading
         a file.
\end{lstlisting}
\caption{Usage message of the confrd program}
\end{figure}


The confrd program can also be compiled on a normal Linux host system. In this case its
only use is to verify the syntax of configuration files. When the confrd program is
invoked without the \texttt{-w} argument it goes through the provided directory
structure and just parses all the files, verifying their syntax. If the \texttt{-p} flag
is specified, then it will also print out a summary of everything it would store to SRAM for
use with the tester module.
\\

The parser understands and implements the configuration formats specified in the previous
section. It was designed with performance in mind, and, as a result, it can parse up to
60000 lines per second on the 100 MHz Nios II system.
\\

In its normal operating mode, with the \texttt{-w} flag, the operation is as follows.
The program will go through each subdirectory in the directory specified on the command
line and look for a \texttt{team.cfg} file in each. First this \texttt{team.cfg} file
is parsed, which contains the address of the Chip Tester backend. Using the information
in this file and the URL to the backend, confrd will first create a result entry in
the database by using the HTTP API. The id of the newly created object, as returned
from the backend, is stored in memory. This id is used to identify subsequent queries
for the same result.
\\

It will then parse each of the vector files in the same directory, one after another.
For each of the vector files it will create a design result entry in the database via
the HTTP API. The returned id is used to identify further HTTP requests storing the
test vector and measurement results.
\\

Most keywords generate a request memory structure for use by the tester hardware
which is written to a SDRAM memory location
of the same size as the SRAM. In the previous configuration example, the following
memory structures/requests would be generated:
\begin{itemize}
 \item The \texttt{team} line in the \texttt{team.cfg} generates a change target request
 \item The \texttt{design} line does not generate any request - it only stores the design name to send it to the backend/database.
 \item The \texttt{frequency} line generates a PLL Reconfiguration request. The factors (multiplication, division, post-loop counter) for a given frequency are taken from a header file
which includes pre-calculated values for each integer frequency between 1 and 100 MHz. These factors are generated by the \texttt{pllfreq.rb} support tool described in a later section.
 \item The \texttt{clock} line generates a DICMD request with the bitmask generated by the individual pins as payload, and the type set to \texttt{DICMD\_SETUP\_MUXES}.
 \item The \texttt{trigger} line generates a DICMD request with the bitmask generated by the individual pins as payload, and the type set to \texttt{DICMD\_TRGMASK}.
 \item The \texttt{pindef} line generates a change bitmask request
 \item Each \texttt{vector} line generates a test vector request
\end{itemize}


Whenever the SRAM staging area is (almost) full, the file has been completely read, or
a \texttt{measure} command appears, the SRAM staging area is written to actual SRAM, followed
by a mem end request (to denote end of the test vectors), and the test module is activated.
confrd then sleeps until the test hardware is done using the driver's \texttt{poll} interface.
As soon as it is done, it resets
the staging area and either continues parsing test vectors or executes the \texttt{measure}
command, as appropriate.
\\

Since the DUT interface holds the last input vector until a new one is processed,
this ensures that \texttt{measure} commands are executed with a known state on the input pins.
\\

Each time the tester, frequency counter or ADC modules complete, their result
is sent to the backend/database via the HTTP API, identified by the design result id
retrieved from the backend at the start of a vector file.
\\

Any syntax or other error that occurs after the \texttt{team.cfg} file has been parsed
will be logged remotely to the backend. All errors and some status messages indicating
the current progress are printed to the character LCD using the DE2 LCD Driver.
\\

The throughput of the complete process is determined by the upload speed. Both the
file parsing and the actual testing on hardware are extremely fast, as is the SRAM
access. Without uploading, 60000 tests complete in roughly 10 seconds. With uploading
this extends to several minutes.
\\

For verification purposes of the test hardware, the confrd program has an additional flag
\texttt{-s} which writes what it would normally write to SRAM into a memory initialization
file that can be used in the \texttt{tester} testbench to initialize the SRAM model.
This allows for verification of the \texttt{tester} module using actual test vectors
as they are generated by the software.
\\

The confrd tool relies on two external libraries; libjansson to encode and decode
JSON structures and libCURL to provide a convenient HTTP interface. Both of these libraries
are used for the HTTP API accesses to the backend.


% on the embedded system!!
%  60000 lines -> 1.28/1.05/0.21
% 120000 lines -> 4.31/3.40/0.84

%PLL generator, etc

\newpage
\section{SPI Flash Programmer and Model}
The SPI Flash Programmer software is able to program the Spansion SPI Flash Memory
on the secondary FPGA board via the the SPI framework present in Linux
since kernel version 2.6.27.

\begin{figure}[h!]
\lstset{basicstyle=\scriptsize\ttfamily}
\begin{lstlisting}
Usage: ./spi_flash [-DsbdlHOLC3] <file to program>
  -D --device   device to use (default /dev/spidev1.1)
  -c            chip erase instead of individual sectors
  -i            print information about flash and exit
  -t            perform the operation(s) on the flash model
  -s --speed    max speed (Hz)
  -d --delay    delay (usec)
  -b --bpw      bits per word
  -l --loop     loopback
  -H --cpha     clock phase
  -O --cpol     clock polarity
  -L --lsb      least significant bit first
  -C --cs-high  chip select active high
  -3 --3wire    SI/SO signals shared
\end{lstlisting}
\caption{Usage message of the spi\_flash program}
\end{figure}

The utility provides three operating modes:

\begin{itemize}
 \item Using the \texttt{-i} flag, the program only prints information about the device such as manufacturer id, unique id and capacity and exits.
 \item Without the \texttt{-i} flag the utility writes the file specified on the command line to the SPI flash by either erasing the complete chip or individual sectors as they are being written.
 \item With the \texttt{-t} flag, all the operations occur on the flash model (described in the next section) instead of on the actual hardware.
\end{itemize}

To achieve these operations, the spi\_flash utility implements support for the SPI Flash commands listed in Table \ref{tab:spi_commands}.

\begin{table}[h!]
\centering
    \begin{tabular}{ | l | p{12cm} |}
    \hline
    Byte & Description \\ \hline
    0x06 & Write Enable Instruction, it is used every time a write or erase instruction
    is performed. It sets the WRITE ENABLE LATCH (WEL) in the status register to 1\\ \hline
    0x04 & Disables the write enable. Sets the WRITE ENABLE LATCH (WEL) in the status register to 0.\\ \hline
    0x50 & Enables the write for the volatile bits in the status register.\\ \hline
    0x05 &  Reads the first status register word \\ \hline
    0x35 &  Reads the second status register word\\ \hline
    0xC7  & Performs a chip erase It sets all the memory within the device to the read state of all 0xFF\\ \hline
    0x03 & Reads from the Flash memory\\ \hline
    0x9F & Reads the JEDEC assigned Manufacturer ID byte and two Device ID
    bytes, Memory Type and Capacity \\ \hline
    0x4B & The Read Unique ID Number instruction accesses a factory-set read-only 64-bit number that is unique to
    each device\\ \hline
    0x90 & Reads the Manufacturer JEDEC ID and the specific device ID\\ \hline
    0x02 & The page program instruction programs a page (256 bytes) of memory \\ \hline
    0xD8 & Erases a  64kB block\\ \hline
    0x52 & Erases a 32kB block\\ \hline
    0x20 & Erases a 4kB sector \\ \hline
    \end{tabular}
    \caption{SPI commands}
    \label{tab:spi_commands}
\end{table}


\subsection{SPI Flash Model}
To verify the correct operation of the SPI Flash Programmer, an SPI Flash
Model was written independently for use with the programmer. It implements the same commands as shown in the
table \ref{tab:spi_commands}.

When in test mode, the spi flash programmer communicates exclusively with the model
instead of the actual kernel SPI framework.

This model was written in order to provide verification for the SPI Flash Programmer
without having the real flash connected.


\newpage
\section{vconfig}
The \texttt{vconfig.sh} script is executed via an \texttt{flock}ed cron job every 10 minutes.
By using \texttt{flock} there will never be more than one instance running at a time, even if
the execution takes longer than 10 minutes.
\\

The \texttt{vconfig.sh} script is used to download virtual design packages from the
backend, program the slave FPGA and run a battery of tests.
\\

The script will look for an SD Card that contains a file called \texttt{vconfig.sh}. This
file needs to contain just one line, giving the address of the Chip Tester backend from
which it can download configurations. The format is as shown below. It differs from the
regular configuration format used for the rest of the system as it is used directly from
a shell script without any parsing - it is only ``sourced''.
\lstset{basicstyle=\scriptsize\ttfamily}
\begin{lstlisting}
BASE_URL="http://192.168.0.10:4567"
\end{lstlisting}

If this file is not present (for example because a different or no SD Card is inserted),
\texttt{vconfig.sh} will not do anything.
\\

If however the file is present and the URL points to a valid ChipTester backend, it will
try to download a virtual design via the HTTP API. If none is available for download it
will exit immediately. Otherwise it will download the file to a temporary location in \texttt{/tmp}.
\\

Once the file is downloaded it will try to decompress trying all of the following archive formats:
\begin{itemize}
 \item ZIP archive
 \item TARball without compression
 \item TARball with gzip compression
 \item TARball with bzip2 compression
 \item TARball with LZMA compression
\end{itemize}

If all attempts fail, the script will give up and log the error remotely to the backend using the
external \texttt{rlog} tool mentioned in the next section.
\\

Next the syntax of the decompressed files is verified by running the \texttt{confrd} tool
without the \texttt{-w} flag. If a syntax error occurs it will be logged remotely and the process
will be aborted.
\\

After the initial validation of the downloaded files, the configuration of the slave FPGA
begins. Using the GPIO interface (described in the previous Drivers section), the \texttt{nCE}
and \texttt{nCONFIG} pins are tied low. Tying \texttt{nCONFIG} low will cause the slave FPGA
to lose all its configuration and enter a reset state, tri-stating all its I/O pins.
\\

While the FPGA is in this reset state, the SPI Flash programmer \texttt{spi\_flash} is used
to write the \texttt{fpga.rbf} contained in the configuration archive to the SPI Configuration
Flash on the slave FPGA board. If the process fails, the error is logged remotely and the
script exits.
\\

If the configuration was written successfully, the \texttt{nCONFIG} pin is asserted, causing
the FPGA to begin reconfiguration from the newly written flash. After waiting 10 seconds
for the FPGA to finish reconfiguration, the FPGA's \texttt{nSTATUS} pin is checked to determine
whether an error occurred during reconfiguration. If so, the error is logged and the operation
is aborted.
\\

Otherwise the \texttt{confrd} program is invoked in its normal mode with the \texttt{-vw} parameter,
causing it to perform all the tests in the provided test cases and logging the results remotely
as explained in the previous section about confrd. The \texttt{-v} flag only changes a flag it sends
to the backend when saving the results - it is only used to mark the result as having been performed
on a ``virtual'' design.


\newpage
\section{Test and support programs and scripts}
\subsection{de2lcd}
The \texttt{de2lcd} program is a standalone program using the DE2 LCD driver to control
the character LCD. It was used to test the driver and controller for the DE2 LCD, but
can also be used as a standalone tool to write to the LCD. It is used from some scripts
to write text to the LCD.


\begin{figure}[h!]
\lstset{basicstyle=\scriptsize\ttfamily}
\begin{lstlisting}
Usage: de2lcd <options>
Valid options are:
 -c
         Clear LCD.
 -n
         Enable blinking cursor.
 -f
         Disable blinking cursor.
 -s <miliseconds>
         Enable left shift every <miliseconds> ms
 -t
         Test.
 -w <Text>
         Writes <Text> starting at first character of LCD
\end{lstlisting}
\caption{Usage message of the de2lcd program}
\end{figure}


\subsection{fcounter}
The \texttt{fcounter} program is a standalone program interfacing with the frequency
counter module via the fcounter driver. It can show the id number of the device,
set the number of timeout cycles (the number of host clock cycles over which the
edges of the input signal are counted), select on which input line to measure the
frequency and enable the frequency counter. After enabling it, it will only exit
after the frequency counter is done. Once it is done, the number of counted edges
can be printed.

\begin{figure}[h!]
\lstset{basicstyle=\scriptsize\ttfamily}
\begin{lstlisting}
Usage: fcounter <options>
Valid options are:
 -m
         Get magic number.
 -e
         Enable.
 -a
         Print edge count.
 -c <cycles>
         Set timeout <cycles>.
 -s <sel>
         Select <sel> line as input.
\end{lstlisting}
\caption{Usage message of the fcounter program}
\end{figure}


\subsection{rlog}
The \texttt{rlog} program is a standalone program that takes the URL of the ChipTester
backend and can log a string remotely to the backend. It is used from several scripts, such
as the \texttt{vconfig.sh} script to log errors remotely. Additionally the log level can be
specified. The level can be one of ``debug'', ``info'', ``warn'' or ``err''.
\begin{figure}[h!]
 \lstset{basicstyle=\scriptsize\ttfamily}
\begin{lstlisting}
Usage: rlog [-l <level>] -b <base_url> <message>
\end{lstlisting}
\caption{Usage message of the rlog program}
\end{figure}


\subsection{umount2}
The \texttt{umount2} tool is a simple implementation around the \texttt{umount} system call.
It was written because the \texttt{umount} tool integrated in busybox failed to unmount
on the Linux 3.3 kernel used for this project. As it was felt that the effort to debug
the problem with the \texttt{umount} in busybox would outweigh the effort of writing the tool
from scratch, the latter approach was chosen.
\\

\texttt{umount2} simply takes either the path of the device that is mounted or the path
a device is mounted on and unmounts it.


\subsection{automount.sh}
The \texttt{automount.sh} tool is invoked by mdev (a busybox tool included in the uClinux
distribution) whenever an SD Card is inserted and removed
from the system. It simply mounts the sd card file system, and if the mounting is
successful and the sd card contains a file named \texttt{chiptester} in its root, it
will execute the \texttt{confrd} program to run the tests contained on the sd card.
\\

On removal of an SD Card, any running instance of the \texttt{confrd} program is killed
and the filesystem unmounted using the aforementioned \texttt{umount2} tool.


\subsection{rc}
The \texttt{rc} script is the main script executed at startup. This script is based
on a default script shipped with uClinux, but with a number of custom modifications.
\\

It sets the hostname, configures the network via DHCP and sets up temporary memory file
systems for several paths such as \texttt{/tmp}, \texttt{/media} and \texttt{/var/log}. The
script also starts the syslogd logging daemon and the cron daemon to execute periodic tasks.
\\

It also sets up the GPIO pins to be accessible, and also sets their direction (whether
they are input or output pins).
\\

As a final step it uses the \texttt{de2lcd} program to write a greeting message to the LCD
and also prints a greeting message to the (serial) console.


\subsection{pllfreq.rb}
The \texttt{pllfreq.rb} tool is a ruby program that, taking a base frequency (i.e. input
frequency to an Altera Cyclone IV PLL), generates a C header file that contains values
for the loop multiplier, divider and post-scale counters to achieve every integer
output frequency between 1MHz and 100 MHz. The output header file is also annotated with the
error between the intended frequency and the actual achieved frequency. The calculation
is done while maintaining valid values for the multiplier and divider so that the VCO
frequency stays in the valid range.
\\

The generated header file is used by the \texttt{confrd} tool to determine which vectors
to pass to the test controller to set the desired frequency in the reconfigurable
clock domain. Figure \ref{listing:pllfreq_h} shows a small excerpt from this file.

\begin{figure}[h!]
\lstset{basicstyle=\scriptsize\ttfamily}
\begin{lstlisting}
       { .m = 32,       .n = 5, .c = 5 },   /* 64 MHz, err: 0.0 */
       { .m = 26,       .n = 5, .c = 4 },   /* 65 MHz, err: 0.0 */
       { .m = 33,       .n = 5, .c = 5 },   /* 66 MHz, err: 0.0 */
       { .m = 47,       .n = 5, .c = 7 },   /* 67 MHz, err: 0.1428571428571388 */
\end{lstlisting}
\caption{Excerpt from the header file generated by the \texttt{pllfreq.rb} script}
\label{listing:pllfreq_h}
\end{figure}