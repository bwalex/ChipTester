\section{\texorpdfstring{\href{mailto:xl1g11@ecs.soton.ac.uk}{Xiaolong Li}} {Xiaolong Li}}


Grade aspect: Since I have four great teammates and they have done excellent jobs, I would say 10 for whole works.

\subsection{Group Formation and Operation}

We are a five-member team: Hornung Alex, Progias Pavlos, Torres Romel, Zhou Bo and me. Alex manages the whole project and I think he just like a leader of us. The main FPGA developed and he does some other interfaces designs. Romel and Pavlos are in charge of the softwares and sever. Zhou is done the PLL parts and some of the hardware design. My job is mostly of the hardware design, especially two PCBs.

To be honest, there may be no disagreements when we determined the roles. Alex should be the leader. Romel and Pavlos are good at software and Zhou and I are more interested in hardware design. So I would say 8 or 9 will be suitable for grading.

Grade your group 	8/10


\subsection{Work Breakdown}

As I mentioned before, some of us are good at software, and some of us are doing well in hardware. And most important, Alex is really a great leader. Regarding to the contribution, honestly Alex may have done more works than us. But I’m not sure about the others. However, I think they all finished their amazing jobs.

In this case, I would give 7 here.

Grade your group’s work breakdown		7/10


\subsection{Planning and Progress}

We got the basic ideas for the project, and discuss them together before we come to the final version. Since my jobs is the PCBs design for the slave FPGA and the interface board, and more important is we all have few idea with the new PCB design tool, Allegro PCB designer, so what I planned is to familiar the tools first. After that we choose the component together, of cause we replace and abandon some of them during the schematic design. By researching the datasheets of them, learning how they work, I pick up the best components and draw the circuit. After finished the schematics with OrCAD capture, I transit it into PCB layout in the Allegro PCB designer. Note that some of the footprint for PCB layout is provided by the manufactures but I create most of them. With all prepared, we sent the Gerber files to the manufacture to print them out.

In this case I have spent lots of time in place and route, so I the 9 will be appropriate.

Grade your group’s planning \& progress	9/10


\subsection{Your Own Contribution}

Most of jobs I have done are the PCB designs as well as some virtual design for this project. As I have mentioned above. The design of the circuit is achieve with the help of all the components’ datasheet and the user manual. The schematic is not really difficult since the circuit has been designed. The most vital part of making a PCB is to create all the footprints for the components in the circuit. Some of them can be found on the website or the providers, but it is not always true. Most of the components footprints need to be drawn, or created by hand. And I think it spent me numbers of days to come up with all of them. Next step is place and route. It seems to be easy but when you did this, it is a tough job to minimize the size and the space and all other factors. Finally, generate the Gerber files which are necessary in printing PCB.

In this case I would say 8 or 9 for this part.

Grade your own contribution	 8/10

\subsection{Reflective Evaluation}

All things are new for me in this project. FPGA, chip tester, and PCB tools. Basically, now I have learned how to design a chip tester. Furthermore, acquire the skill of group work and the procedure of designing a project that are quite useful for my future.

Currently, many designers would like to use Altium PCB tools to design. Therefore, using the Allegro tools seems to be a very good experience to make PCBs.